\chapter{Overview of traditional GSM networks}

\section{What is GSM?}
GSM, or Global System for Mobile Communications,  is a European standard for the Mobile telecommunications and it is considered as one of the most popular standard worldwide.There are thirteen different frequency bands de�ned in GSM. However, the 850 MHz, 900 MHz, 1800 MHz, and 1900 MHz bands are the most commonly used. The frequency bands employed within each of the four ranges are similarly organized. They differ essentially only in the frequencies, such that various synergy effects can be taken advantage of; hence we only give some details for the usage in the 900 MHz band.


In the 900 MHz band, a total of 70 MHz bandwidth is allocated, two 25-MHz frequency
bands for uplink and downlink and a 20 MHz unused guard band between them. The MS transmits
in the 890- to 915-MHz range (uplink) and the BTS transmits in the 935- to 960-MHz
(downlink) band. This corresponds to 124 duplex channels, where each channel within a BTS is
referred to as an Absolute Radio Frequency Channel Number (ARFCN). This number describes a pair of frequencies, one uplink and one downlink, and is given a channel index between C0 and C123, with C0 designated as the beacon channel. An ARFCN could be used to calculate the exact frequency (in MHz) of the radio channel. In the GSM 900 band, this is computed by the following equations:

\begin{align}
F_{uplink}(n) &= 890 + 0.2*n \qquad & 1\leq{}n\leq{}124 \nonumber\\
F_{downlink}(n) &=F_{uplink}(n) + 45  \qquad & 1\leq{}n\leq{}124 \nonumber
\end{align}

Similar formulas are also de�ned for the other GSM frequency bands.


The principle component groups of a GSM network are as follows:
\begin{itemize}
	\item The Mobile Station (MS)
	\item The Base Station System (BSS)
	\item The Network Switching System(NSS)
\end{itemize}

Diagram of GSM architecture
fig


\subsection{Mobile Station}
The MS consists of two parts, the Mobile Equipment (ME) and  Subscriber Identity module (SIM). The ME has an identity number called the International Mobile Equipment Identity (IMEI) associated with it, which is unique for that particular device and permanently stored in it.The SIM card consists the International Mobile Subscriber Identity(IMSI) number which is used to identify the subscriber to the system. The IMEI and IMSI are independent of each other and hence, allowing personal mobility.

\subsection{Base Station System}
The GSM Base Station System is the equipment located at a cell site. It comprises a combination of digital and RF equipment. The BSS provides the link between the MS and the Mobile Services Switching Centre (MSC).
The BSS consists mainly of:
\begin{description}
	\item[The Base Transceiver Station(BTS)]  
	The BTS contains the RF components that provide the air interface for a particular cell. This is the part of the GSM network which communicates with the MS. The antenna is included as part of the BTS.
	\item[The Base Station Controller(BSC)]  
	The BSC  provides the control for the BSS. The BSC communicates directly with the MSC. The BSC may control single or multiple BTSs. Crucial functions like radio channel link establishment, frequency hopping, and handovers from one cell to another.

\end{description}

\subsection{Network Switching Subsystem} The Network Switching System includes the main switching functions of the GSM network. It also contains the databases required for subscriber data and mobility management. Its main function is to manage communications between the GSM network and other telecommunications networks. The main components of the Network Switching System are: 
\begin{description}
	\item[Mobile Services Switching Centre(MSC)] The MSC does call-switching and its overall purpose is the same as that of any telephone exchange. When the MSC provides the interface between the PSTN and the BSSs in the GSM network it will be known as a Gateway MSC. In this position it will provide the switching required for all MS originated or terminated traffic. Each MSC provides service to MSs located within a defined geographic coverage area. One MSC is capable of supporting a regional capital with approximately one million inhabitants. 
The functions carried out by the MSC are: Call Processing, Operations and Maintenance Support, Internetwork Interworking and Billing
	\item[Home Location Register (HLR)] The HLR is a central database that contains details of each mobile phone subscriber that is authorized to use the GSM core network.The  IMSI�s of each SIMs act as primary key to each HLR record. Each MSISDN is also a primary key to the HLR record. The HLR data is stored for as long as a subscriber remains with the mobile phone operator.Data stored in HLR against each IMSI are, GSM services that the subscriber has requested ,GPRS settings to allow subscriber to access packet services,current location of subscriber etc
	\item[Visitor Location Register (VLR)] The VLR is a database of the subscribers who have roamed into the jurisdiction of the MSC which it serves. Each main base station in the network is served by exactly one VLR, hence a subscriber cannot be present in more than one VLR at a time.The data stored in the VLR has either been received from the HLR, or collected from the MS.Data stored include: IMSI (the subscriber's identity number), authentication data,MSISDN ,GSM services that the subscriber is allowed to access, The HLR address of the subscriber.

\end{description}

\section{Um Interface}
The Um interface is the air interface for the GSM mobile telephone standard. It is the interface between the MS and the BTS. It is called Um because it is the mobile analog to the U interface of ISDN. Um is defined in the GSM 04.xx and 05.xx series of specifications.


The layers of GSM are initially defined in GSM 04.01 Section 7 and roughly follow the OSI model. Um is defined in the lower three layers of the model.

\subsection{Physical Layer (L1)}
The Um physical layer is defined in the GSM 05.xx series of specifications, with the introduction and overview in GSM 05.01. For most channels, Um L1 transmits and receives 184-bit control frames or 260-bit vocoder frames over the radio interface in 148-bit bursts with one burst per timeslot. There are three sublayers:

\begin{description}
	\item[Radiomodem] This is the actual radio transceiver. GSM uses 8PSK modulation with 1 bit per symbol which produces a 13/48 MHz (270.833 kHz or 270.833 K symbols/second) symbol rate and a channel spacing of 200 kHz. Since adjacent channels overlap, the standard does not allow adjacent channels to be used in the same cell. The standard defines several bands ranging from 400 MHz to 1990 MHz.GSM is frequency duplexed, meaning that the network and MS transmit on different frequencies, allowing the BTS to transmit and receive at the same time. Transmission from the network to the MS is called �downlink" and from the MS to the network is called �uplink". GSM uplink and downlink bands are separated by 45 or 50 MHz. Uplink/downlink channel pairs are identified by an index called the ARFCN. Within the BTS, these ARFCNs are given arbitrary carrier indexes C0..Cn-1, with C0 designated as a Beacon Channel and always operated at constant power. The radio channel is time-multiplexed into 8 timeslots, each with a duration of 156.25 symbol periods. These 8 timeslots form a frame of 1,250 symbol periods. The capacity associated with a single timeslot on a single ARFCN is called a physical channel (PCH) and referred to as �CnTm" where n is a carrier index and m is a timeslot index (0-7).Each timeslot is occupied by a radio burst with a guard interval, two payload fields, tail bits, and a midamble.
	
	\item[Multiplexing and Timing] GSM uses TDMA to subdivide each radio channel into as many as 16 traffic channels or as many as 64 control channels. The multiplexing patterns are defined in GSM 05.02.Each physical channel is time-multiplexed into multiple logical channels according to the rules of GSM 05.02. Traffic channel multiplexing follows a 26-frame (0.12 second) cycle called a "multiframe". Control channels follow a 51-frame multiframe cycle. The C0T0 physical channel carries the synchronization channel(SCH), which encodes the timing state of the BTS to facilitate synchronization to the TDMA pattern.
	
	\item[FEC Coding] The coding sublayer provides forward error correction. As a general rule, each GSM channel uses a block parity code (usually a Fire code), a rate-1/2, 4th-order convolutional code and a 4-burst or 8-burst interleaver.

\end{description}

\subsection{Data Link Layer(L2)}
The Um data link layer, LAPDm, is defined in GSM 04.05 and 04.06. LAPDm is the mobile analog to ISDN's LAPD.

\subsection{Network layer(L3)}
The Um network layer is defined in GSM 04.07 and 04.08 and has three sublayers. A subscriber terminal must establish a connection in each sublayer before accessing the next higher sublayer.

\begin{description}
	\item[Radio Resource (RR)] This sublayer manages the assignment and release of logical channels on the radio.

	\item[Mobility Management (MM)] This sublayer authenticates users and tracks their movements from cell to cell. It is normally terminated in the VLR or HLR.

	\item[Call Control (CC)] This sublayer connects telephone calls and is taken directly from ITU-T Q.931. GSM 04.08 Annex E provides a table of corresponding paragraphs in GSM 04.08 and ITU-T Q.931 along with a summary of differences between the two. The CC sublayer is terminated in the MSC.

\end{description}

The access order is RR, MM, CC. The release order is the reverse of that. Note that none of these sublayers terminate in the BTS itself. The standard GSM BTS operates only in layers 1 and 2.
