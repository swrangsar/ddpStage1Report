\chapter{Introduction}
\section{Background}
The electromagnetic radio spectrum is a natural resource that remains underutilized \cite{haykin05}. It is licensed by governments for use by transmitters and receivers. With the explosive proliferation of cell phones and other wireless communication devices, we cannot afford to waste our spectral resources anymore.

In November 2002, the Spectrum Policy Task Force, a group under the Federal Communications Commission(FCC) in the United States, published a report saying \cite{repFCC}, 
\begin{quote}
``In many bands, spectrum access is a more significant problem than physical scarcity of spectrum, in large part due to legacy command-and-control regulation that limits the ability of potential spectrum users to obtain such access.''
\end{quote}

If we were to scan the radio spectrum even in metropolitan places where it's heavily used, we would find that \cite{staple04} :
\begin{enumerate}
	\item some frequency bands are unoccupied most of the time,
	\item some are only partially occupied and
	\item the rest are heavily used.
\end{enumerate}

The underutilization of spectral resources leads us to think in terms of \emph{spectrum holes}, which is defined as \cite{kolodzy01}:
\begin{quote}
\emph{A spectrum hole is a band of frequencies assigned to a primary user, but, at a particular time and specific geographic location, the band is not being utilized by that user.
}
\end{quote}

Spectrum utilization can be improved significantly by allowing a secondary user (who is not being serviced) to use a spectrum hole unoccupied by the primary user at the right location and the time in question \cite{haykin05}. \emph{Cognitive Radio}, which is usually implemented using a software defined radio, has been proposed as the means to promote the efficient use of spectral resources.

\section{Cognitive Radio}



\section{Organization}
