\chapter{Introduction}
\section{Background}
The electromagnetic radio spectrum is a natural resource that remains underutilized \cite{haykin05}. It is licensed by governments for use by transmitters and receivers. With the explosive proliferation of cell phones and other wireless communication devices, we cannot afford to waste our spectral resources anymore.

In November 2002, the Spectrum Policy Task Force, a group under the Federal Communications Commission(FCC) in the United States, published a report saying \cite{repFCC}, 
\begin{quote}
``In many bands, spectrum access is a more significant problem than physical scarcity of spectrum, in large part due to legacy command-and-control regulation that limits the ability of potential spectrum users to obtain such access.''
\end{quote}

If we were to scan the radio spectrum even in metropolitan places where it's heavily used, we would find that \cite{staple04} :
\begin{enumerate}
	\item some frequency bands are unoccupied most of the time,
	\item some are only partially occupied and
	\item the rest are heavily used.
\end{enumerate}

\section{Cognitive Radio}
A cognitive radio is an intelligent radio that can be programmed and configured dynamically. Its transceiver is designed to use the best wireless channels in its vicinity. Such a radio automatically detects available channels in wireless spectrum, then accordingly changes its transmission or reception parameters to allow more concurrent wireless communications in a given spectrum band at one location. This process is a form of dynamic spectrum management\cite{wikiCR}.

\section{Motivation}
Studies have shown that most of the spectrum allotted to licensed networks remain unused most of the time\cite{repFCC}. To utilize these unused spectral resources we can make use of dynamic spectrum management. We can allow secondary (unlicensed) users to utilize the spectrum whenever that particular spectrum becomes available. For this we need cognitive capabilities to sense the availability of the spectrum.


\section{Organization}