\begin{abstract}

The concept of interference temperature was introduced by the FCC as a new metric for quantifying and managing interference. Using this model, cognitive radios (CRs) operating in licensed frequency bands would be able to measure their current interference environment and adjust their transmission characteristics so as not to raise the interference temperature over a regulatory limit.

As of now it is hard to predict whether interference temperature is going to be practical because no one has been able to come up with a solution that is agreed upon by everyone. The research on interference temperature was abandoned by the Federal Communications Commission (FCC) in 2007 but there is some hope of a comeback.

This document highlights what I have learned about interference temperature as part of my supervised research exposition.

\end{abstract}
